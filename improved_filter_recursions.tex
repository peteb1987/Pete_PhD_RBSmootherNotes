\documentclass{article}

\usepackage{amsmath}
\usepackage{IEEEtrantools}
\usepackage{color}

\newenvironment{meta}[0]{\color{red} \em}{}

\title{Rao-Blackwellised Particle Smoothing: Incorporating suggested improvements from Fredrik Lindsten and Gerlach et al.}
\author{Pete Bunch}
\date{25th September 2012}

\begin{document}

\maketitle

\section{Basics}

Our model is,
%
\begin{IEEEeqnarray}{rCl}
  u_t &\sim& p(u_t | u_{t-1}) \nonumber \\
  z_t &=& A(u_{t}) z_{t-1} + q_{t} \nonumber \\
  y_t &=& C(u_t) z_{t} + r_{t} \\
  q_{t} &\sim& \mathcal{N}(\cdot|0,Q(u_t)) \nonumber \\
  r_{t} &\sim& \mathcal{N}(\cdot|0,R(u_t)) \nonumber
\end{IEEEeqnarray}

This is the more standard model, not the one used in our conference paper.

With smoothing, we would like to estimate
%
\begin{equation}
p(z_{1:T}, u_{1:T}|y_{1:T}) = p(z_{1:T}| u_{1:T}, y_{1:T}) p(u_{1:T}|y_{1:T}),
\end{equation}

where $z_{1:T}$ and $u_{1:T}$ are the linear and nonlinear parts of the state respectively.

\section{Backward Simulation}

The nonlinear smoothing distribution can be factorised in the following way,
%
\begin{equation}
p(u_{1:T}|y_{1:T}) = p(u_T|y_{1:T}) \prod_{t=1}^{T-1} p(u_t|u_{t+1:T}, y_{1:T})     ,
\end{equation}

which suggests a backwards sequential sampling approach. The factors may be expanded as,
%
\begin{IEEEeqnarray}{rCl}
p(u_t|u_{t+1:T}, y_{1:T}) & = & \int p(u_{1:t}, z_t|u_{t+1:T}, y_{1:T}) dz_t du_{1:t-1} \nonumber \\
 & \propto & \int p(y_{t+1:T}|z_t, u_{t+1:T}) p(u_{1:t}, z_t| u_{t+1:T}, y_{1:t}) dz_t du_{1:t-1} \nonumber \\
 & \propto & \int p(y_{t+1:T}|z_t, u_{t+1:T}) p(u_{t+1:T}|u_{t}) p(z_t | u_{1:t}, y_{1:t}) p(u_{1:t}|y_{1:t}) dz_t du_{1:t-1} \nonumber \\
 & \propto & \int \left[ \int p(z_t|u_{1:t}, y_{1:t}) p(y_{t+1:T}|z_t, u_{t+1:T}) dz_t \right] p(u_{1:t}|y_{1:t}) p(u_{t+1}|u_{t}) du_{1:t-1}     .  \IEEEeqnarraynumspace
\end{IEEEeqnarray}

Running a Rao-Blackwellised particle filter gives us an estimate of,
%
\begin{equation}
p(u_{1:t}|y_{1:t}) \approx \sum_i w_t^{(i)} \delta_{u_{1:t}^{(i)}}(u_{1:t})     .
\end{equation}

Substituting this in, particles may be drawn from the smoothing distribution by sequentially sampling,
%
\begin{IEEEeqnarray}{rCl}
p(u_t|u_{t+1:T}, y_{1:T}) & \approx & \int \sum_i w_t^{(i)} \delta_{u_{1:t}^{(i)}}(u_{1:t}) p(u_{t+1}|u_{t}) \int p(z_t|u_{1:t}, y_{1:t}) p(y_{t+1:T}|z_t, u_{t+1:T}) dz_t du_{1:t-1} \nonumber \\
 & = & \sum_i w_t^{(i)} p(u_{t+1}|u_{t}^{(i)}) \int p(z_t|u_{1:t}^{(i)}, y_{1:t}) p(y_{t+1:T}|z_t, u_{t+1:T}) dz_t \: \delta_{u_{t}^{(i)}}(u_{t}) \nonumber \\
 & = & \sum_i \rho^{(i)} \delta_{u_{t}^{(i)}}(u_{t})     ,
\end{IEEEeqnarray}

where,
%
\begin{equation}
\rho^{(i)} \propto w_t^{(i)} p(u_{t+1}|u_{t}^{(i)}) \int p(z_t|u_{1:t}^{(i)}, y_{1:t}) p(y_{t+1:T}|z_t, u_{t+1:T}) dz_t     .
\end{equation}

\section{The Linear State Integral}

\subsection{Original Formulation}

The first term in the integral is given by the forward Kalman filter estimate from the forward stage
%
\begin{equation}
p(z_t|u_{1:t}^{(i)}, y_{1:t}) = \mathcal{N}( z_t|m_t^{(i)}, P_t^{(i)} )     .
\end{equation}

The second term is given by a backward filter in the style of the two-filter smoother,
%
\begin{equation}
p(y_{t+1:T}|z_t, u_{t+1:T}) = \hat{Z}_{t} \mathcal{N}( z_{t}|\hat{m}_t^{b}, \hat{P}_t^{b} )     .
\end{equation}

The integral is then given by,
%
\begin{IEEEeqnarray}{rCl}
\IEEEeqnarraymulticol{3}{l}{\int p(z_t|u_{1:t}^{(i)}, y_{1:t}) p(y_{t+1:T}|z_t, u_{t+1:T}) dz_t } \nonumber \\
\qquad \qquad & = & \hat{Z}_{t} \int \mathcal{N}( z_t|m_t^{(i)}, P_t^{(i)} ) \mathcal{N}( z_{t}|\hat{m}_t^{b}, \hat{P}_t^{b} ) dz_t \nonumber \\
 & = & \hat{Z}_{t} \mathcal{N}( m_{t}^{(i)}|\hat{m}_t^{b}, P_t^{(i)} + \hat{P}_t^{b} )     .
\end{IEEEeqnarray}

The normalisation factor $\hat{Z}_{t}$ is constant over the sampling index $i$, and may thus be forgotten about (comes out in the normalisation) as long as it is finite. This will be the case apart from during the initialisation and when $A(u_t)$ is not invertible.

It only remains to calculate the backward filter estimate. These are given by a recursion,
%
\begin{IEEEeqnarray}{rCl}
p(y_{t:T}|z_t, u_{t:T}) & = & Z_t \mathcal{N}( z_t|m_t^{b}, P_t^{b} ) \\
p(y_{t+1:T}|z_t, u_{t+1:T}) & = & \hat{Z}_t \mathcal{N}( z_t|\hat{m}_t^{b}, \hat{P}_t^{b} )     .
\end{IEEEeqnarray}

These are standard. We use the shorthand $A_t = A(u_t)$, $Q_t = Q(u_t)$, etc.

Predict:

\begin{IEEEeqnarray}{rCl}
\IEEEeqnarraymulticol{3}{l}{ p(y_{t+1:T}|z_t, u_{t+1:T}) } \nonumber \\
\qquad \qquad & = & \int p(y_{t+1:T}|z_{t+1}, u_{t+1:T}) p(z_{t+1}|u_{t+1}, z_t) dz_{t+1} \nonumber \\
 & = & Z_{t+1} \int \mathcal{N}( z_{t+1}|m_{t+1}^{b}, P_{t+1}^{b} ) \mathcal{N}( z_{t+1}|A_t z_t, Q_{t+1} ) \nonumber \\
 & = & \hat{Z}_{t} \mathcal{N}( z_t | \hat{m}_{t}^{b}, \hat{P}_t^{b} )
\end{IEEEeqnarray}

Thus:

\begin{IEEEeqnarray}{rCl}
\hat{m}_t^{b} & = & A_t^{-1} m_{t+1}^{b} \\
\hat{P}_t^{b} & = & A_t^{-1} (P_{t+1}^{b}+Q_{t+1}) A_t^{-T} \\
\hat{Z}_t & = & Z_{t+1} \left| \left| A_t \right| \right|^{-1}
\end{IEEEeqnarray}

Update:

\begin{IEEEeqnarray}{rCl}
\IEEEeqnarraymulticol{3}{l}{ p(y_{t:T}|z_t, u_{t:T}) } \nonumber \\
\qquad \qquad & = & p(y_{t+1:T}|z_t, u_{t+1:T}) p(y_t|z_t, u_t) \nonumber \\
 & = & \hat{Z}_{t} \mathcal{N}( z_t|\hat{m}_t^{b}, \hat{P}_t^{b} ) \mathcal{N}( y_t|C_t z_t, R_t ) \nonumber \\
 & = & Z_t \mathcal{N}( z_t|m_t^{b}, P_t^{b} )
\end{IEEEeqnarray}

where:

\begin{IEEEeqnarray}{rCl}
S_t & = & C_t \hat{P}_t^{b} C_t^T + R_t \\
K_t & = & \hat{P}_t^{b} C_t^T S_t^{-1} \\
m_t^{b} & = & \hat{m}_t^{b} + K_t (y_t - C_t \hat{m}_t^{b}) \\
P_t^{b} & = & \hat{S}_t - K_t S_t K_t^T \\
Z_t & = & \hat{Z}_t \mathcal{N}( y_t|C_t \hat{m}_t, S_t )
\end{IEEEeqnarray}

This formulation has the disadvantage that it fails when $A_t$ is not invertible. Also the initialisation is messy.

\subsection{Better Formulation}

We use the same Kalman filter for the forward estimate
%
\begin{equation}
p(z_t|u_{1:t}^{(i)}, y_{1:t}) = \mathcal{N}( z_t|m_t^{(i)}, P_t^{(i)} )     .
\end{equation}

For the backward estimate, assume the following form,
%
\begin{equation}
p(y_{t+1:T}|z_t, u_{t+1:T}) = \hat{Z}_{t} \exp\left\{ -\frac{1}{2} \left[ z_t^T \hat{\Omega}_t z_t - 2 \hat{\lambda}_t^T z_t \right] \right\}     .
\end{equation}

We will need the following identity (repeatedly),
%
\begin{IEEEeqnarray}{rCl}
 \int \exp\left\{ -\frac{1}{2} \left[ z^T \Upsilon z - 2 \psi^T z + c \right] \right\} dz = \sqrt{ \left|\left| \Upsilon^{-1} \right|\right| } \exp\left\{ -\frac{1}{2} \left[ c - \psi^T \Upsilon^{-1} \psi \right] \right\}     .
\end{IEEEeqnarray}

In addition, we will be using the integral trick for rank-deficient Gaussians,
%
\begin{IEEEeqnarray}{rCl}
 \mathcal{N}(z|\mu,\Sigma) & = & \int \delta_{\mu + \Delta \epsilon}(z) \mathcal{N}(\epsilon|0,I) d\epsilon     ,
\end{IEEEeqnarray}

where $\Sigma = \Delta \Delta^T$. If $\Sigma$ is $d \times d$ and has rank $r$, then $\Delta$ is $d \times r$ and $\epsilon$ has length $r$. Together, these two identities constitute Fredrik's Lemma 1.

The integral is given by,
%
\begin{IEEEeqnarray}{rCl}
\IEEEeqnarraymulticol{3}{l}{\int p(z_t|u_{1:t}^{(i)}, y_{1:t}) p(y_{t+1:T}|z_t, u_{t+1:T}) dz_t } \nonumber \\
\qquad \qquad & = & \hat{Z}_{t} \int \mathcal{N}( z_t|m_t^{(i)}, P_t^{(i)} ) \exp\left\{ -\frac{1}{2} \left[ z_t^T \hat{\Omega}_t z_t - 2 \hat{\lambda}_t^T z_t \right] \right\} dz_t \nonumber \\
 & = & \hat{Z}_{t} \left| \left| 2 \pi I \right| \right|^{-\frac{1}{2}} \int \exp\left\{ -\frac{1}{2} \eta \right\} d\epsilon \nonumber \\
 & = & \hat{Z}_{t} \left| \left| 2 \pi I \right| \right|^{-\frac{1}{2}} \left| \left| 2 \pi \left( \Gamma_t^{(i)T} \hat{\Omega}_t \Gamma_t^{(i)} + I \right)^{-1} \right| \right|^{\frac{1}{2}} \exp\left\{ -\frac{1}{2} \nu \right\} d\epsilon \nonumber \\
 & = & \hat{Z}_{t} \left| \left| \left( \Gamma_t^{(i)T} \hat{\Omega}_t \Gamma_t^{(i)} + I \right) \right| \right|^{-\frac{1}{2}} \exp\left\{ -\frac{1}{2} \nu \right\} \nonumber \\
\end{IEEEeqnarray}

where $P_t^{(i)} = \Gamma_t^{(i)} \Gamma_t^{(i)T}$ and,
%
\begin{IEEEeqnarray}{rCl}
 \eta  & = &  \left(m_t^{(i)} + \Gamma_t^{(i)} \epsilon\right)^T \hat{\Omega}_t \left(m_t^{(i)} + \Gamma_t^{(i)} \epsilon\right) - 2 \hat{\lambda}_t^T \left(m_t^{(i)} + \Gamma_t^{(i)} \epsilon\right) + \epsilon^T \epsilon \nonumber \\
       & = & \epsilon^T \left[ \Gamma_t^{(i)T} \hat{\Omega}_t \Gamma_t^{(i)} + I \right] - 2 \left[ \hat{\lambda}_t^T - m_t^{(i)T} \hat{\Omega}_t \Gamma_t^{(i)} \right] \epsilon + \left[ m_t^{(i)T} \hat{\Omega}_t m_t^{(i)} - \hat{\lambda}_t^T m_t^{(i)} \right]     ,
\end{IEEEeqnarray}

and
%
\begin{IEEEeqnarray}{rCl}
 \nu & = & \left[ m_t^{(i)T} \hat{\Omega}_t m_t^{(i)} - \hat{\lambda}_t^T m_t^{(i)} \right] \nonumber \\
     &   & \qquad - \left[ \hat{\lambda}_t^T - m_t^{(i)T} \hat{\Omega}_t \Gamma_t^{(i)} \right] \left[ \Gamma_t^{(i)T} \hat{\Omega}_t \Gamma_t^{(i)} + I \right]^{-1} \left[ \hat{\lambda}_t - \Gamma_t^{(i)T} \hat{\Omega}_t m_t^{(i)} \right]
\end{IEEEeqnarray}

Now we need backwards recursions for the following,
%
\begin{IEEEeqnarray}{rCl}
p(y_{t:T}|z_t, u_{t:T}) = Z_{t} \exp\left\{ -\frac{1}{2} \left[ z_t^T \Omega_t z_t - 2 \lambda_t^T z_t \right] \right\} \\
p(y_{t+1:T}|z_t, u_{t+1:T}) = \hat{Z}_{t} \exp\left\{ -\frac{1}{2} \left[ z_t^T \hat{\Omega}_t z_t - 2 \hat{\lambda}_t^T z_t \right] \right\}     .
\end{IEEEeqnarray}

Assume we have estimates of $Z_{t+1}$, $\lambda_{t+1}$ and $\Omega_{t+1}$.

Predict:

\begin{IEEEeqnarray}{rCl}
 \IEEEeqnarraymulticol{3}{l}{ p(y_{t+1:T}|z_t, u_{t+1:T}) = \int p(y_{t+1:T}|z_{t+1}, u_{t+1:T}) p(z_{t+1} | z_t, u_{t+1}) dz_{t+1} } \nonumber \\
 & = & Z_{t+1} \int \exp\left\{ -\frac{1}{2} \left[ z_{t+1}^T \Omega_{t+1} z_{t+1} - 2 \lambda_{t+1}^T z_{t+1} \right] \right\} \mathcal{N}(z_{t+1}|A_{t+1} z_t, Q_{t+1}) dz_{t+1} \nonumber \\
 & = & Z_{t+1} \left|\left| 2 \pi I \right|\right|^{-\frac{1}{2}} \int \exp\left\{ -\frac{1}{2} \zeta \right\} d\epsilon \nonumber \\
 & = & Z_{t+1} \left|\left| \left( F_{t+1}^T \Omega_{t+1} F_{t+1} + I \right) \right|\right|^{-\frac{1}{2}} \exp\left\{ -\frac{1}{2} \xi \right\}
\end{IEEEeqnarray}

where $Q_{t+1} = F_{t+1} F_{t+1}^T$ and,
%
\begin{IEEEeqnarray}{rCl}
 \zeta & = &  \left(A_{t+1} z_t + F_{t+1} \epsilon\right)^T \Omega_{t+1} \left(A_{t+1} z_t + F_{t+1} \epsilon\right) - 2 \lambda_{t+1}^T \left(A_{t+1} z_t + F_{t+1} \epsilon\right) + \epsilon^T \epsilon \nonumber \\
       & = & \epsilon^T \left[ F_{t+1}^T \Omega_{t+1} F_{t+1} + I \right] \epsilon - 2 \left[ \lambda_{t+1}^T F_{t+1} - z_t^T A_{t+1}^T \Omega_{t+1} F_{t+1} \right] \epsilon \nonumber \\
       &   & \qquad \qquad \qquad \qquad \qquad + \: \left[ z_t^T A_{t+1}^T \Omega_{t+1} A_{t+1} z_t - 2 \lambda_{t+1}^T A_{t+1} z_t \right]     ,
\end{IEEEeqnarray}

and,
%
\begin{IEEEeqnarray}{rCl}
 \xi & = & \left[ z_t^T A_{t+1}^T \Omega_{t+1} A_{t+1} z_t - \lambda_{t+1}^T A_{t+1} z_t \right] \nonumber \\
     &   & \qquad - \left[ \lambda_{t+1}^T F_{t+1} - z_t^T A_{t+1}^T \Omega_{t+1} F_{t+1} \right] \left[ F_{t+1}^T \Omega_{t+1} F_{t+1} + I \right]^{-1} \left[ F_{t+1}^T \lambda_{t+1} - F_{t+1}^T \Omega_{t+1} A_{t+1} z_t \right] \nonumber \\
     & = & z_t^T \left[ A_{t+1}^T \left( \Omega_{t+1} - \Omega_{t+1} F_{t+1} \left( F_{t+1}^T \Omega_{t+1} F_{t+1} + I \right)^{-1} F_{t+1}^T \Omega_{t+1} \right) A_{t+1} \right] z_t \nonumber \\
     &   & - \: 2 \left[ \lambda_{t+1}^T \left( I - F_{t+1} \left( F_{t+1}^T \Omega_{t+1} F_{t+1} + I \right)^{-1} F_{t+1}^T \Omega_{t+1} \right) A_{t+1} \right] z_t \nonumber \\
     &   & + \: \left[ - \lambda_{t+1}^T F_{t+1} \left( F_{t+1}^T \Omega_{t+1} F_{t+1} + I \right)^{-1} F_{t+1}^T \lambda_{t+1} \right]     .
\end{IEEEeqnarray}

Hence the prediction recursions are,
%
\begin{IEEEeqnarray}{rCl}
 \hat{\Omega}_t  & = & A_{t+1}^T \left( \Omega_{t+1} - \Omega_{t+1} F_{t+1} \left( F_{t+1}^T \Omega_{t+1} F_{t+1} + I \right)^{-1} F_{t+1}^T \Omega_{t+1} \right) A_{t+1} \\
 \hat{\lambda}_t & = & A_{t+1}^T \left( I - F_{t+1} \left( F_{t+1}^T \Omega_{t+1} F_{t+1} + I \right)^{-1} F_{t+1}^T \Omega_{t+1} \right) \lambda_{t+1} \\
 \hat{Z}_t       & = & Z_{t+1} \left|\left| \left( F_{t+1}^T \Omega_{t+1} F_{t+1} + I \right) \right|\right|^{-\frac{1}{2}} \nonumber \\
                 &   & \qquad \qquad \qquad \times \exp\left\{ \frac{1}{2} \lambda_{t+1}^T F_{t+1} \left( F_{t+1}^T \Omega_{t+1} F_{t+1} + I \right)^{-1} F_{t+1}^T \lambda_{t+1} \right\}     .
\end{IEEEeqnarray}

Update:
%
\begin{IEEEeqnarray}{rCl}
 \IEEEeqnarraymulticol{3}{l}{ p(y_{t:T}|z_t, u_{t:T}) = p(y_{t+1:T}|z_t, u_{t+1:T}) p(y_{t}|z_t, u_{t}) } \nonumber \\
 & = & \hat{Z}_t \left|\left| 2 \pi R_{t} \right|\right|^{-\frac{1}{2}} \exp\left\{ -\frac{1}{2} \omega \right\}     ,
\end{IEEEeqnarray}

where,
%
\begin{IEEEeqnarray}{rCl}
 \omega & = & z_t^T \hat{\Omega}_t z_t - 2 \hat{\lambda}_t^T z_t + (y_t - C_t z_t)^T R_t^{-1} (y_t - C_t z_t) \nonumber \\
        & = & z_t^T \left[ \hat{\Omega}_t + C_t^T R_t^{-1} C_t \right] z_t - 2 \left[ \hat{\lambda}_t^T + y_t^T C_t^T R_t^{-1} \right] z_t + \left[ y_t^T R_t^{-1} y_t \right]     .
\end{IEEEeqnarray}

Hence the update recursions are,
%
\begin{IEEEeqnarray}{rCl}
 \Omega_t  & = & \hat{\Omega}_t + C_t^T R_t^{-1} C_t \\
 \lambda_t & = & \hat{\lambda}_t + R_t^{-1} C_t y_t \\
 Z_t       & = & \hat{Z}_t \left|\left| 2 \pi R_{t} \right|\right|^{-\frac{1}{2}} \exp\left\{ -\frac{1}{2} y_t^T R_t^{-1} y_t \right\}      .
\end{IEEEeqnarray}

These recursions work even when $A_t$, $\Omega_{t+1}$, or $Q_{t+1}$ is not invertible, provided $R_t$ is full rank. Moreover they are easy to initialise,
%
\begin{IEEEeqnarray}{rCl}
 p(y_T | z_T, u_T) & = & \mathcal{N}(y_T | C_T z_T, R_T) \nonumber \\
                   & = & \left|\left| 2 \pi R_{T} \right|\right|^{-\frac{1}{2}} \exp\left\{ -\frac{1}{2} (y_T - C_T z_T)^T R_T^{-1} (y_T - C_T z_T) \right\} \nonumber \\
                   & = & \left|\left| 2 \pi R_{T} \right|\right|^{-\frac{1}{2}} \exp\left\{ -\frac{1}{2} y_T R_T^{-1} y_T \right\} \exp\left\{ z_T^T C_T^T R_T^{-1} C_T z_T - 2 (y_T^T R_T^{-1} C_T) z_T) \right\}     .
\end{IEEEeqnarray}

Hence,
\begin{IEEEeqnarray}{rCl}
 \Omega_T  & = & C_T^T R_T^{-1} C_T \nonumber \\
 \lambda_T & = & C_T^T R_T^{-1} y_T \nonumber \\
 Z_T       & = & \left|\left| 2 \pi R_{T} \right|\right|^{-\frac{1}{2}} \exp\left\{ -\frac{1}{2} y_T R_T^{-1} y_T \right\}     .
\end{IEEEeqnarray}

$\Omega_T$ is not invertible, but this doesn't matter, unlike the original formulation.

Because the normalisation constants are always finite (for full rank $R_t$) and depend only on the future nonlinear states, we need never calculate them.



\section{Mixed Linear/Nonlinear Models}

The extension to mixed models is straightforward. The new model is,
%
\begin{IEEEeqnarray}{rCl}
  u_{t+1} &=& A_u(u_t) z_t + q_{u,t} \nonumber \\
  z_{t+1} &=& A_z(u_t) z_t + q_{z,t} \nonumber \\
  y_t &=& C(u_t) z_{t} + r_{t}. \\
  q_{u,t} &\sim& \mathcal{N}(\cdot|0,Q_u(u_t)) \nonumber \\
  q_{z,t} &\sim& \mathcal{N}(\cdot|0,Q_z(u_t)) \nonumber \\
  r_{t} &\sim& \mathcal{N}(r_{t}|0,R(u_t)) \nonumber     .
\end{IEEEeqnarray}

We use equivalent short-hands for the matrices to those used previously. The thing to watch out for here is that $u_{t+1}$ is not independent of the past observations when conditioned on $u_t$. Furthermore, in order to predict $z_t$ from future observations, we will need $u_t$, unlike before. Finally, $z_{t+1}$ will be less useful as an auxiliary variable, because it will not ``block'' the sequence of state variables (because $u_{t+1}$ depends directly on $z_t$). The following derivations proceeds essentially as before.

Start with the backwards conditional distribution,
%
\begin{IEEEeqnarray}{rCl}
p(u_t|u_{t+1:T}, y_{1:T}) & = & \int p(u_{1:t}, z_t|u_{t+1:T}, y_{1:T}) dz_t du_{1:t-1} \nonumber \\
 & \propto & \int p(y_{t+1:T}, u_{t+1:T}|z_t, u_{t}) p(u_{1:t}, z_t| y_{1:t}) dz_t du_{1:t-1} \nonumber \\
 & \propto & \int \left[ \int p(y_{t+1:T}, u_{t+1:T}|z_t, u_{t}) p(z_t | u_{1:t}, y_{1:t}) dz_t \right] p(u_{1:t}|y_{1:t}) du_{1:t-1} \IEEEeqnarraynumspace     ,
\end{IEEEeqnarray}

This expression has the same form as the hierarchical model except for the exclusion of the nonlinear transition density and the modified backwards filter term. Hence, the weights have a very similar form to that derived in the previous section. We can use the particle filter and forward Kalman filter approximations as before. It only remains to estimate $p(y_{t+1:T}, u_{t+1:T}|z_t, u_{t})$.

We need backwards recursions for the following,
%
\begin{IEEEeqnarray}{rCl}
p(y_{t:T}, u_{t+1:T}|z_t, u_t) & = & Z_{t} \exp\left\{ -\frac{1}{2} \left[ z_t^T \Omega_t z_t - 2 \lambda_t^T z_t \right] \right\} \\
p(y_{t+1:T}, u_{t+1:T}|z_t, u_t) & = & \hat{Z}_{t} \exp\left\{ -\frac{1}{2} \left[ z_t^T \hat{\Omega}_t z_t - 2 \hat{\lambda}_t^T z_t \right] \right\}     .
\end{IEEEeqnarray}

Assume we have estimates of $Z_{t+1}$, $\lambda_{t+1}$ and $\Omega_{t+1}$.

Predict:
%
\begin{IEEEeqnarray}{rCl}
 \IEEEeqnarraymulticol{3}{l}{ p(y_{t+1:T}, u_{t+1:T}|z_t, u_t) = \int p(y_{t+1:T}, u_{t+2:T}|z_{t+1}, u_{t+1}) p(z_{t+1}, u_{t+1} | z_t, u_{t}) dz_{t+1}} \nonumber \\
 & = & \int p(y_{t+1:T}, u_{t+2:T}|z_{t+1}, u_{t+1}) p(z_{t+1} | z_t, u_{t}), p(u_{t+1} | z_t, u_{t}) dz_{t+1} \nonumber \\
 & = & Z_{t+1} \left|\left| 2 \pi Q_{u,t} \right|\right|^{-\frac{1}{2}} \left|\left| 2 \pi I \right|\right|^{-\frac{1}{2}} \int \exp\left\{ -\frac{1}{2} \zeta \right\} d\epsilon \\
 & = & Z_{t+1} \left|\left| 2 \pi Q_{u,t} \right|\right|^{-\frac{1}{2}} \left|\left| \left( F_{z,t}^T \Omega_{t+1} F_{z,t} + I \right) \right|\right|^{-\frac{1}{2}} \exp\left\{ -\frac{1}{2} \xi \right\}     ,
\end{IEEEeqnarray}

where $Q_{z,t} = F_{z,t} F_{z,t}^T$ and,
%
\begin{IEEEeqnarray}{rCl}
 \zeta & = &  \left(A_{z,t} z_t + F_{z,t} \epsilon\right)^T \Omega_{t+1} \left(A_{z,t} z_t + F_{z,t} \epsilon\right) \nonumber \\
       &   & - \: 2 \lambda_{t+1}^T \left(A_{z,t} z_t + F_{z,t} \epsilon\right) + \epsilon^T \epsilon \nonumber \\
       &   & + \: (u_{t+1} - A_{u,t} z_{t})^T Q_{u,t}^{-1} (u_{t+1} - A_{u,t} z_{t}) \nonumber \\
       & = & \epsilon^T \left[ F_{z,t}^T \Omega_{t+1} F_{z,t} + I \right] \epsilon \nonumber \\
       &   & - \: 2 \left[ \lambda_{t+1}^T F_{z,t} - z_t^T A_{z,t}^T \Omega_{t+1} F_{z,t} \right] \epsilon \nonumber \\
       &   & + \: \left[ z_t^T A_{z,t}^T \Omega_{t+1} A_{z,t} z_t - 2 \lambda_{t+1}^T A_{z,t} z_t + (u_{t+1} - A_{u,t} z_{t})^T Q_{u,t}^{-1} (u_{t+1} - A_{u,t} z_{t}) \right]     ,
\end{IEEEeqnarray}

and,
%
\begin{IEEEeqnarray}{rCl}
 \xi & = & z_t^T A_{z,t}^T \Omega_{t+1} A_{z,t} z_t - 2 \lambda_{t+1}^T A_{z,t} z_t + (u_{t+1} - A_{u,t} z_{t})^T Q_{u,t}^{-1} (u_{t+1} - A_{u,t} z_{t}) \nonumber \\
     &   & - \: \left[ \lambda_{t+1}^T F_{z,t} - z_t^T A_{z,t}^T \Omega_{t+1} F_{z,t} \right] \left[ F_{z,t}^T \Omega_{t+1} F_{z,t} + I \right]^{-1} \left[ F_{z,t} \lambda_{t+1} - F_{z,t}^T \Omega_{t+1} A_{z,t} z_t \right] \nonumber \\
     & = & z_t^T \left[ A_{z,t}^T \left( \Omega_{t+1} - \Omega_{t+1} F_{z,t} \left( F_{z,t}^T \Omega_{t+1} F_{z,t} + I \right)^{-1} F_{z,t}^T \Omega_{t+1} \right) A_{z,t} + A_{u,t}^T Q_{u,t}^{-1} A_{u,t} \right] z_t \nonumber \\
     &   & - \: 2 \left[ \lambda_{t+1}^T \left( I - F_{z,t} \left( F_{z,t}^T \Omega_{t+1} F_{z,t} + I \right)^{-1} F_{z,t}^T \Omega_{t+1} \right) A_{z,t} + u_{t+1}^T Q_{u,t}^{-1} A_{u,t} \right] z_t \nonumber \\
     &   & + \: \left[ - \lambda_{t+1}^T F_{z,t} \left( F_{z,t}^T \Omega_{t+1} F_{z,t} + I \right)^{-1} F_{z,t}^T \lambda_{t+1} + u_{t+1}^T Q_{u,t}^{-1} u_{t+1} \right]     .
\end{IEEEeqnarray}

Hence the prediction recursions are,
%
\begin{IEEEeqnarray}{rCl}
 \hat{\Omega}_t  & = & A_{z,t}^T \left( \Omega_{t+1} - \Omega_{t+1} F_{z,t} \left( F_{z,t}^T \Omega_{t+1} F_{z,t} + I \right)^{-1} F_{z,t}^T \Omega_{t+1} \right) A_{z,t} + A_{u,t}^T Q_{u,t}^{-1} A_{u,t} \\
 \hat{\lambda}_t & = & A_{z,t}^T \left( I - F_{z,t} \left( F_{z,t}^T \Omega_{t+1} F_{z,t} + I \right)^{-1} F_{z,t}^T \Omega_{t+1} \right) \lambda_{t+1} + A_{u,t}^T Q_{u,t}^{-1} u_{t+1} \\
 \hat{Z}_t       & = & Z_{t+1} \left|\left| 2 \pi Q_{u,t} \right|\right|^{-\frac{1}{2}} \left|\left| \left( F_{z,t}^T \Omega_{t+1} F_{z,t} + I \right) \right|\right|^{-\frac{1}{2}} \nonumber \\
                 &   & \times \exp\left\{ -\frac{1}{2} \left[ - \lambda_{t+1}^T F_{z,t} \left( F_{z,t}^T \Omega_{t+1} F_{z,t} + I \right)^{-1} F_{z,t}^T \lambda_{t+1} + u_{t+1}^T Q_{u,t}^{-1} u_{t+1} \right] \right\}     .
\end{IEEEeqnarray}

Note that unlike hierarchical model, these all depend on $u_t$ through the matrices $A_{z,t}$, $A_{u,t}$, $Q_{z,t}$, $Q_{u,t}$, so we will need to run a backward filter for every particle in the filter distribution (or at least, every one which we propose if we use rejection or MH sampling).

Update (this is identical to the hierarchical model):
%
\begin{IEEEeqnarray}{rCl}
 \IEEEeqnarraymulticol{3}{l}{ p(y_{t:T}, u_{t+1:T}|z_t, u_t) = p(y_{t+1:T}, u_{t+1:T}|z_t, u_t) p(y_{t}|z_t, u_{t}) } \nonumber \\
 & = & \hat{Z}_t \left|\left| 2 \pi R_{t} \right|\right|^{-\frac{1}{2}} \exp\left\{ -\frac{1}{2} \omega \right\}     ,
\end{IEEEeqnarray}

where,
%
\begin{IEEEeqnarray}{rCl}
 \omega & = & z_t^T \hat{\Omega}_t z_t - 2 \hat{\lambda}_t^T z_t + (y_t - C_t z_t)^T R_t^{-1} (y_t - C_t z_t) \nonumber \\
        & = & z_t^T \left[ \hat{\Omega}_t + C_t^T R_t^{-1} C_t \right] z_t - 2 \left[ \hat{\lambda}_t^T + y_t^T C_t^T R_t^{-1} \right] z_t + \left[ y_t^T R_t^{-1} y_t \right]     .
\end{IEEEeqnarray}

Hence the update recursions are,
%
\begin{IEEEeqnarray}{rCl}
 \Omega_t  & = & \hat{\Omega}_t + C_t^T R_t^{-1} C_t \\
 \lambda_t & = & \hat{\lambda}_t + R_t^{-1} C_t y_t \\
 Z_t       & = & \hat{Z}_t \left|\left| 2 \pi R_{t} \right|\right|^{-\frac{1}{2}} \exp\left\{ -\frac{1}{2} y_t^T R_t^{-1} y_t \right\}      .
\end{IEEEeqnarray}

Finally, the initialisation. This requires the observation that when $t=T$, $p(y_{t:T}, u_{t+1:T} | z_t, u_t) = p(y_T | z_T, u_T)$. But then it's identical to the hierarchical model.
%
\begin{IEEEeqnarray}{rCl}
 p(y_T | z_T, u_T) & = & \mathcal{N}(y_T | C_T z_T, R_T) \nonumber \\
                   & = & \left|\left| 2 \pi R_{T} \right|\right|^{-\frac{1}{2}} \exp\left\{ -\frac{1}{2} (y_T - C_T z_T)^T R_T^{-1} (y_T - C_T z_T) \right\} \nonumber \\
                   & = & \left|\left| 2 \pi R_{T} \right|\right|^{-\frac{1}{2}} \exp\left\{ -\frac{1}{2} y_T R_T^{-1} y_T \right\} \exp\left\{ z_T^T C_T^T R_T^{-1} C_T z_T - 2 (y_T^T R_T^{-1} C_T) z_T) \right\}     .
\end{IEEEeqnarray}

Hence,
\begin{IEEEeqnarray}{rCl}
 \Omega_T  & = & C_T^T R_T^{-1} C_T \nonumber \\
 \lambda_T & = & C_T^T R_T^{-1} y_T \nonumber \\
 Z_T       & = & \left|\left| 2 \pi R_{T} \right|\right|^{-\frac{1}{2}} \exp\left\{ -\frac{1}{2} y_T R_T^{-1} y_T \right\}     .
\end{IEEEeqnarray}

\section{Further Generalisations}

I think from here it's a simple step to generalise to the ``full model'' in [Sch\"on et al. 2005] in which the equations for both $z_{t+1}$ and $u_{t+1}$ also contain a nonlinear function of $u_t$. This is, after all, just a deterministic shift in the mean when conditioned on $u_t$.

More interestingly is the case when $q_{z,t}$ and $q_{u,t}$ are correlated, which we examine here,
%
\begin{IEEEeqnarray}{rCl}
  u_{t+1} &=& A_u(u_t) z_t + q_{u,t} \nonumber \\
  z_{t+1} &=& A_z(u_t) z_t + q_{z,t} \nonumber \\
  q_{t} & = & \begin{bmatrix}q_{z,t} \\ q_{u,t} \end{bmatrix} \sim \mathcal{N}(\cdot|0,Q_t)     .
\end{IEEEeqnarray}

First partition $Q_t$,
%
\begin{IEEEeqnarray}{rCl}
 Q_t & = & \begin{bmatrix} Q_{z,t} & Q_{C,t} \\ Q_{C,t}^T & Q_{u,t} \end{bmatrix}      ,
\end{IEEEeqnarray}

in the obvious way. Now factorise it,
%
\begin{IEEEeqnarray}{rCl}
 Q_t & = & F_t F_t^T     ,
\end{IEEEeqnarray}

and partition the factors as well,
%
\begin{IEEEeqnarray}{rCl}
 F_t & = & \begin{bmatrix} F_{z,t} \\ F_{u,t} \end{bmatrix}    ,
\end{IEEEeqnarray}

such that we now have,
%
\begin{IEEEeqnarray}{rCl}
 Q_t & = & \begin{bmatrix} F_{z,t} F_{z,t}^T & F_{z,t} F_{u,t}^T \\ F_{u,t} F_{z,t}^T & F_{u,t} F_{u,t}^T \end{bmatrix}     .
\end{IEEEeqnarray}

Now we can write the joint density of $z_{t+1}, u_{t+1} | z_{t}, u_{t}$ using the integral trick (no assumptions yet about the invertibility of $Q_t$ or its sub-matrices),
%
\begin{IEEEeqnarray}{rCl}
 p(z_{t+1}, u_{t+1} | z_{t}, u_{t}) & = & \int \delta_{A_{z,t} z_t + F_{z,t} \epsilon}(z_{t+1}) \delta_{A_{u,t} z_t + F_{u,t} \epsilon}(u_{t+1}) \mathcal{N}(\epsilon|0,I) d\epsilon     .
\end{IEEEeqnarray}

Substituting this into the prediction equation integral gives us,
%
\begin{IEEEeqnarray}{rCl}
 \IEEEeqnarraymulticol{3}{l}{ p(y_{t+1:T}, u_{t+1:T}|z_t, u_t) = \int p(y_{t+1:T}, u_{t+2:T}|z_{t+1}, u_{t+1}) p(z_{t+1}, u_{t+1} | z_t, u_{t}) dz_{t+1}} \nonumber \\
 & = & Z_{t+1}  \int \int \exp\left\{ -\frac{1}{2} \left[ z_{t+1}^T \Omega_{t+1} z_{t+1} - 2 \lambda_{t+1}^T z_{t+1} \right] \right\} \delta_{A_{z,t} z_t + F_{z,t} \epsilon}(z_{t+1}) \delta_{A_{u,t} z_t + F_{u,t} \epsilon}(u_{t+1}) \mathcal{N}(\epsilon|0,I) d\epsilon dz_{t+1} \nonumber \\
  & = & Z_{t+1} \left| \left| 2 \pi I \right| \right|^{-\frac{1}{2}} \int \exp\left\{ -\frac{1}{2} \zeta \right\} \delta_{A_{u,t} z_t + F_{u,t} \epsilon}(u_{t+1}) d\epsilon \nonumber \\
  & = & Z_{t+1} \left| \left| 2 \pi I \right| \right|^{-\frac{1}{2}} \left| \left| 2 \pi \left(F_{z,t}^T \Omega_{t+1} F_{z,t} + I\right)^{-1} \right| \right|^{\frac{1}{2}} \left| \left| 2 \pi \left(F_{u,t} \left(F_{z,t}^T \Omega_{t+1} F_{z,t} + I\right)^{-1} F_{u,t}^T\right) \right| \right|^{-\frac{1}{2}} \exp\left\{ -\frac{1}{2} \xi \right\} \nonumber \\
  & = & Z_{t+1} \left| \left| 2 \pi I \right| \right|^{-\frac{1}{2}} \frac{ \left| \left| \Psi_t \right| \right|^{\frac{1}{2}} }{ \left| \left|\left(F_{u,t} \Psi_t F_{u,t}^T\right) \right| \right|^{\frac{1}{2}} }     \exp\left\{ -\frac{1}{2} \xi \right\}
\end{IEEEeqnarray}

where,
%
\begin{IEEEeqnarray}{rCl}
 \zeta & = & \left(A_{z,t} z_t + F_{z,t} \epsilon\right)^T \Omega_{t+1} \left(A_{z,t} z_t + F_{z,t} \epsilon\right) - 2 \lambda_{t+1}^T \left(A_{z,t} z_t + F_{z,t} \epsilon\right) + \epsilon^T \epsilon \nonumber \\
       & = & \epsilon^T \left[ F_{z,t}^T \Omega_{t+1} F_{z,t} + I \right] \epsilon \nonumber \\
       &   & - \: 2 \left[ \lambda_{t+1}^T F_{z,t} - z_t^T A_{z,t}^T \Omega_{t+1} F_{z,t} \right] \epsilon \nonumber \\
       &   & + \: \left[ z_t^T A_{z,t}^T \Omega_{t+1} A_{z,t} z_t - 2 \lambda_{t+1}^T A_{z,t} z_t \right]     ,
\end{IEEEeqnarray}

\begin{IEEEeqnarray}{rCl}
 \xi & = & \left[ z_t^T A_{z,t}^T \Omega_{t+1} A_{z,t} z_t - 2 \lambda_{t+1}^T A_{z,t} z_t \right] \nonumber \\  % 2 terms
     &   & - \: \left[ \lambda_{t+1}^T F_{z,t} - z_t^T A_{z,t}^T \Omega_{t+1} F_{z,t} \right] \Psi_t \left[ F_{z,t}^T \lambda_{t+1} - F_{z,t}^T \Omega_{t+1} A_{z,t} z_t \right] \nonumber \\  % 3 terms
     &   & + \: \left[ u_{t+1} - A_{u,t} z_t - F_{u,t} \Psi_t \left( F_{z,t}^T \lambda_{t+1} - F_{z,t}^T \Omega_{t+1} A_{z,t} z_t \right) \right]^T \nonumber \\ % 3 terms
     &   & \qquad \times \left(F_{u,t} \Psi_t F_{u,t}^T\right)^{-1} \left[ u_{t+1} - A_{u,t} z_t - F_{u,t} \Psi_t \left( F_{z,t}^T \lambda_{t+1} - F_{z,t}^T \Omega_{t+1} A_{z,t} z_t \right) \right] \nonumber \\
     & = & \left[ z_t^T A_{z,t}^T \Omega_{t+1} A_{z,t} z_t - 2 \lambda_{t+1}^T A_{z,t} z_t \right] \nonumber \\  % 2 terms
     &   & - \: \left[ \lambda_{t+1}^T F_{z,t} - z_t^T A_{z,t}^T \Omega_{t+1} F_{z,t} \right] \Psi_t \left[ F_{z,t}^T \lambda_{t+1} - F_{z,t}^T \Omega_{t+1} A_{z,t} z_t \right] \nonumber \\  % 3 terms
     &   & + \: \left[ u_{t+1} - F_{u,t} \Psi_t F_{z,t}^T \lambda_{t+1} + \left(F_{u,t} \Psi_t F_{z,t}^T \Omega_{t+1} A_{z,t} - A_{u,t} \right) z_t \right]^T \nonumber \\ % 3 terms
     &   & \qquad \times \left(F_{u,t} \Psi_t F_{u,t}^T\right)^{-1} \left[ u_{t+1} - F_{u,t} \Psi_t F_{z,t}^T \lambda_{t+1} + \left(F_{u,t} \Psi_t F_{z,t}^T \Omega_{t+1} A_{z,t} - A_{u,t} \right) z_t \right] \nonumber \\
     & = & z_t^T \left[ A_{z,t}^T \Omega_{t+1} A_{z,t} - A_{z,t}^T \Omega_{t+1} F_{z,t} \Psi_t F_{z,t}^T \Omega_{t+1} A_{z,t} + \gamma_t^T \left(F_{u,t} \Psi_t F_{u,t}^T\right)^{-1} \gamma_t \right] z_t \nonumber \\ % 3 terms
     &   & - 2 \left[ \lambda_{t+1}^T A_{z,t} - \lambda_{t+1}^T F_{z,t} \Psi_t F_{z,t}^T \Omega_{t+1} A_{z,t} - \left[ u_{t+1}^T - \lambda_{t+1}^T F_{z,t} \Psi_t F_{u,t}^T \right] \left(F_{u,t} \Psi_t F_{u,t}^T\right)^{-1} \gamma_t \right] z_t \nonumber \\  % 3 terms
     &   & + \left[ - \lambda_{t+1}^T F_{z,t} \Psi_t F_{z,t}^T \lambda_{t+1} + \left[ u_{t+1}^T - \lambda_{t+1}^T F_{z,t} \Psi_t F_{u,t}^T \right] \left(F_{u,t} \Psi_t F_{u,t}^T\right)^{-1} \left[ u_{t+1} - F_{u,t} \Psi_t F_{z,t}^T \lambda_{t+1} \right] \right]      , %
\end{IEEEeqnarray}

\begin{IEEEeqnarray}{rCl}
 \Psi_t   & = & \left( F_{z,t}^T \Omega_{t+1} F_{z,t} + I \right)^{-1} \nonumber \\
 \gamma_t & = & \left(F_{u,t} \Psi_t F_{z,t}^T \Omega_{t+1} A_{z,t} - A_{u,t} \right)     .
\end{IEEEeqnarray}

Hence the recursions are,
%
\begin{IEEEeqnarray}{rCl}
 \hat{\Omega}_t & = & A_{z,t}^T \Omega_{t+1} A_{z,t} - A_{z,t}^T \Omega_{t+1} F_{z,t} \Psi_t F_{z,t}^T \Omega_{t+1} A_{z,t} + \gamma_t^T \left(F_{u,t} \Psi_t F_{u,t}^T\right)^{-1} \gamma_t \nonumber \\
 \hat{\lambda}_t & = & A_{z,t}^T \lambda_{t+1} - A_{z,t}^T \Omega_{t+1} F_{z,t} \Psi_t F_{z,t}^T \lambda_{t+1} - \gamma_t^T \left(F_{u,t} \Psi_t F_{u,t}^T\right)^{-1} \left[ u_{t+1} - F_{u,t} \Psi_t F_{z,t}^T \lambda_{t+1} \right] \nonumber \\
 \hat{Z}_t & = & Z_{t+1} \left| \left| 2 \pi I \right| \right|^{-\frac{1}{2}} \frac{ \left| \left| \Psi_t \right| \right|^{\frac{1}{2}} }{ \left| \left|\left(F_{u,t} \Psi_t F_{u,t}^T\right) \right| \right|^{\frac{1}{2}} }     \nonumber \\
                 &   & \times \exp\left\{ -\frac{1}{2} \left[ - \lambda_{t+1}^T F_{z,t} \Psi_t F_{z,t}^T \lambda_{t+1} + \left[ u_{t+1}^T - \lambda_{t+1}^T F_{z,t} \Psi_t F_{u,t}^T \right] \left(F_{u,t} \Psi_t F_{u,t}^T\right)^{-1} \left[ u_{t+1} - F_{u,t} \Psi_t F_{z,t}^T \lambda_{t+1} \right] \right] \right\}
\end{IEEEeqnarray}

The last line follows from normalising the exponential and expressing the delta function as a limit of a Gaussian density, then using standard Gaussian formulas. The result looks horrible, and its quite likely that I've made an algebra mistake.

For the recursions to work, we require that $\left(F_{u,t} \Psi_t F_{u,t}^T\right)$ be invertible. Lets consider when this happens.

Denote the length of $q_t$, $q_{u,t}$ and $q_{z,t}$ by $d_t$, $d_{u,t}$ and $d_{z,t}$ respectively, and the ranks of $Q_t$, $Q_{u,t}$ and $Q_{z,t}$ by $r_t$, $r_{u,t}$ and $r_{z,t}$ respectively. $\left(F_{u,t} \Psi_t F_{u,t}^T\right)$ is invertible if $F_{u,t}$ is full rank. $F_{u,t}$ is a $d_{u,t} \times r_t$ matrix, so the condition is satisfied provided $d_{u,t} \leq r_t$ (which is highly probable --- otherwise $Q_{u,t}$ would have to have a nullity at least as great as the rank of $Q_{z,t}$) and $\text{rank}\left(F_{u,t}\right) = d_{u,t}$. These conditions are satisfied if and only if $Q_{u,t}$ is invertible.

%If $q_{u,t}$ and $q_{z,t}$ are independent, then we would find $r_{z,t}$ columns of zeros in $F_{u,t}$, so $\text{rank}\left(F_{u,t}\right) \leq r_t - r_{z,t} = r_{u,t}$. Since $r_{u,t} \leq d_{u,t}$, the conditions are satisfied if and only if $d_{u,t} = r_{u,t}$. In words, $Q_{u,t}$ must be full rank and hence invertible, as we previously discovered.

We can do a quick check of the recursions by examining the case where $Q_{C,t} = 0$. In this case, the following simplifications can be made,
%
\begin{IEEEeqnarray}{rCl}
 F_{z,t} & = & \begin{bmatrix} \tilde{F}_{z,t} & 0 \end{bmatrix} \nonumber \\
 F_{u,t} & = & \begin{bmatrix} 0 & \tilde{F}_{u,t} \end{bmatrix} \nonumber \\
 \Psi_t  & = & \begin{bmatrix} \left(\tilde{F}_{z,t} \Omega_{t+1} \tilde{F}_{z,t}^T + I\right)^{-1} & 0 \\ 0 & I \end{bmatrix} \nonumber \\
 F_{z,t} \Psi_t F_{z,t}^T & = & \tilde{F}_{z,t} \left(\tilde{F}_{z,t} \Omega_{t+1} \tilde{F}_{z,t}^T + I\right)^{-1} \tilde{F}_{z,t}^T \nonumber \\
 F_{u,t} \Psi_t F_{u,t}^T & = & \tilde{F}_{u,t} \tilde{F}_{u,t}^T = Q_{u,t} \nonumber \\
 F_{u,t} \Psi_t F_{z,t}^T & = & 0 \nonumber \\
 \gamma_t & = & -A_{u,t} \nonumber     .
\end{IEEEeqnarray}

Using these, it can be shown that the recursions simplify to those derived for the non-correlated case.

\end{document} 